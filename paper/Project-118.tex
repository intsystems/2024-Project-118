\documentclass[a4paper, 12pt]{article} %{article}
\usepackage{arxiv}

\usepackage[utf8]{inputenc}
\usepackage[english, russian]{babel}
\usepackage[T2A]{fontenc}
\usepackage{url}
\usepackage{booktabs}
\usepackage{amsfonts}
\usepackage{amsmath}
\usepackage{nicefrac}
\usepackage{microtype}
\usepackage{lipsum}
\usepackage{graphicx}
\usepackage{natbib}
\usepackage{doi}
\renewcommand{\abstractname}{Аннотация}

\newcommand\argmin{\mathop{\arg\min}}
\newcommand{\T}{^{\text{\tiny\sffamily\upshape\mdseries T}}}
\newcommand{\hchi}{\hat{\boldsymbol{\chi}}}
\newcommand{\hphi}{\hat{\boldsymbol{\varphi}}}
\newcommand{\bchi}{\boldsymbol{\chi}}
\newcommand{\A}{\mathbf{A}}
\newcommand{\bb}{\mathbf{b}}
\newcommand{\B}{\mathcal{B}}
\newcommand{\W}{\mathbf{W}}
\newcommand{\E}{\mathbf{E}}
\newcommand{\V}{\mathbb{V}}
\renewcommand{\U}{\mathbb{U}}
\newcommand{\x}{\mathbf{x}}
\newcommand{\y}{\mathbf{y}}
\newcommand{\Y}{\mathbf{Y}}
\newcommand{\X}{\mathbf{X}}
\newcommand{\Z}{\mathbf{Z}}
\newcommand{\hx}{\hat{x}}
\newcommand{\hX}{\hat{\X}}
\newcommand{\hy}{\hat{y}}
\newcommand{\M}{\mathcal{M}}
\newcommand{\N}{\mathcal{N}}
\newcommand{\R}{\mathbb{R}}
\newcommand{\p}{p(\cdot)}
\newcommand{\cc}{\mathbf{c}}
\newcommand{\m}{\mathbf{m}}
\newcommand{\bt}{\mathbf{t}}
\newcommand{\e}{\mathbf{e}}
\newcommand{\h}{\mathbf{h}}
\newcommand{\q}{q(\cdot)}
\newcommand{\uu}{\mathbf{u}}
\newcommand{\vv}{\mathbf{v}}
\newcommand{\dd}{\partial}


\title{Непрерывное время при построении нейроинтерфейса BCI.}

\author{ Соболевский Ф.\,А.\\
	Кафедра интеллектуальных систем \\
        ФПМИ МФТИ\\ 
	\texttt{sobolevskii.fa@phystech.edu} 
	\AND
        Консультант: Самохина А.\,М.\\
	Кафедра интеллектуальных систем \\
        ФПМИ МФТИ\\
	\texttt{alina.samokhina@phystech.edu} 
        \AND
        Эксперт: д.ф.-м.н. Стрижов В.\,В.\\
	Вычислительный центр им. А.\,А.\,Дородницына\\
        ФИЦ ИУ РАН \\
	\texttt{strijov@phystech.edu}
	%% Coauthor \\
	%% Affiliation \\
	%% Address \\
	%% \texttt{email} \\
	%% \And
	%% Coauthor \\
	%% Affiliation \\
	%% Address \\
	%% \texttt{email} \\
	%% \And
	%% Coauthor \\
	%% Affiliation \\
	%% Address \\
	%% \texttt{email} \\
}
\date{\today}

%\renewcommand{\shorttitle}{\textit{arXiv} Template}

%%% Add PDF metadata to help others organize their library
%%% Once the PDF is generated, you can check the metadata with
%%% $ pdfinfo template.pdf
\hypersetup{
pdftitle={A template for the arxiv style},
pdfsubject={q-bio.NC, q-bio.QM},
pdfauthor={David S.~Hippocampus, Elias D.~Striatum},
pdfkeywords={First keyword, Second keyword, More},
}

\begin{document}
\maketitle


\begin{abstract}
В задачах декодирования сигналов входные данные представляют собой одномерные или многомерные временные ряды. Применение методов, основанных на нейронных обыкновенных дифференциальных уравнениях, позволяет работать с временными рядами как с непрерывными по времени. Подобные методы могут давать заметно более точные по метрикам качества результаты в задачах классификации сигналов, чем методы, работающие с дискретным представлением временных рядов. В данной работе рассматриваются различные методы, основанные на непрерывном представлении временных рядов, в приложении к задаче классификации электроэнцефалограмм (ЭЭГ) и аппроксимации исходного сигнала. Предложенный в работе метод предполагает работу с тензорным представлением сигнала. Основной результат работы – построение модели, работающей с непрерывным по времени тензорным представлением сигнала и анализ эффективности данного метода в сравнении с современными методами обработки сигнала, использующими его дискретное представление.
\end{abstract}


\keywords{ЭЭГ \and непрерывные временные ряды \and нейронные ОДУ \and тензорное представление сигнала}


\section{Введение}
Данная работа посвящена декодированию сигналов. Примерами актуальных задач декодирования сигналов являются задачи расшифровки сигналов акселерометров, пульсометров, электроэнцефалограмм (ЭЭГ) и электрокортикограмм (ЭКоГ). Многие сигналы, с которыми приходится работать в практических задачах, по природе своей непрерывны по времени, однако распространенный подход к расшифровке сигнала~--- работа с ним, как с дискретным временным рядом. Основной проблемой при применении такого подхода является нерегулярность данных: данные могут иметь неравные отсчёты по времени или пропуски, что несколько усложняет их обработку.

В ряде недавних работ по обработке временных рядов были предложены новые методы, позволяющие работать с временными рядами как с непрерывными по времени. Таковыми являются, например, методы, основанные на применении \textit{нейронных обыкновенных дифференциальных уравнениях} (НОДУ) в качестве скрытого слоя рекурентной нейронной сети \cite{neural_ode}. В работе \cite{cde} был предложен подход, основанный на применении \textit{нейронных управляемых дифференциальных уравнений}, устраняющий однозначную зависимость решения дифференциального уравнения от начальных данных и корректирующий предсказания нейросети на основе данных, поступающих после исходного предсказания. Другие методы, работающие с непрерывным представлением сигналов, включают в себя нейронные стохастические дифференциальные уравнения \cite{neural_sde}, алгоритм S4 \cite{algo_s4} и другие.

Предполагается, что подходы, работающие с непрерывными по времени данными, устраняют проблемы, связанные с нерегулярностью данных. В некоторых задачах декодирования сигналов методы, основанные на НОДУ, и другие методы, работающие с непрерывным представлением сигнала, показали большую точность, чем методы декодирования дискретных последовательностей \cite{neural_ode}, \cite{cde}. В данной работе фокусом являются задачи классификации сигналов головного по признаку наличия в них отклика на некоторый стимул.

В качестве основной задачи декодирования сигнала в данной работе рассматривается классификация ЭЭГ. В последнее время большое количество работ посвящено методам считывания мозговой активности и декодирования информации \cite{Hu2018,Song2017,Loza2017,Eliseyev2016,Gaglianese2016,Bundy2016,Morishita2014, goncharenko}. Основным приложением данных методов являются нейрокомпьютерные интерфейсы (НКИ)~--- технология, позволяющая человеку взаимодействовать с компьютером с помощью анализа данных о мозговой активности.

Целью данной работы является исследование алгоритмов, основанных на нейронных дифференциальных уравнениях (НДУ), в задаче декодирования ЭЭГ для последующего восстановления непрерывной зависимости движения конечности, и сравнение точности по стандартной метрике MSE с результатами, полученными методами, работающими с дискретным представлением времени. Для этого указанные модели, наряду с современными моделями, работающими с дискретным представлением времени, применяются на выборке P300, содержащей данных ЭЭГ и сведения о наличии импульса P300 в сигнале \cite{p300}, с целью достижения
высокой точности декодирования сигнала и оптимальной сложности модели по
количеству параметров.

Применение методов, основанных на НДУ, позволяет работать с ЭЭГ как с непрерывными по времени, что предоставляет возможности дальнейшего исследования непрерывных сигналов, например, в контексте построения фазовых пространств для непрерывного времени и/или пространства. Применимость вышеуказанных методов для решения актуальных задач декодирвания сигналов может послужить мотивацией для развития теории моделей машинного обучения, работающих с непрерывными объектными пространствами.


\section{Постановка задачи}
    В задаче классификации сигнала рассматриваются два варианта входных данных: исходные данные, являющиеся регулярными по сетке времени, и изменённые данные, содержащие 80\% отсчётов времени, выбранных случайным образом для каждого элемента сигнала. 
    
    Для каждой задачи выборка делится на две части: обучающую и валидационную в соотношении 80:20.
    
\subsection{Выборка P300}
\paragraph{Регулярные данные.}
    
Выборка изначально является регулярной по времени. Пусть, в выборке присутствует $M$ наблюдений. Тогда сигнал и целевая переменная определяются следующим образом:
    
    $$\X = \{\X_i\}_{i=1}^{M},$$
    $$\X_i = \{\x_t\}_{t\in T}, \ \x_t \in \R^E, \ T = \{t_i\}_{i=1}^{N}, t_i \in \R,$$ 
    
    $$\Y = \{y_i\}_{i=1}^{M},\ y_i \in \{0, 1\},$$
    
    $$\Delta t = t_i - t_{i-1} = \text{const},$$
    $$E=8 \text{~--- количество электродов,}$$
    $$N=40 \text{~--- количество наблюдений в одном отрезке ЭЭГ.}$$
    
    \paragraph{Нерегулярные данные.}
    
    Определение сигнала и целевой переменной для нерегулярных данных совпадает с определением для регулярных, за исключением длины сигнала и шага по времени:
    $$\Delta t = t_i - t_{i-1} \neq \text{const}$$
    $$E=8 \text{--- количество электродов}$$
    $$N=32 \text{--- количество наблюдений в одном отрезке ЭЭГ}$$

    \paragraph{Постановка задачи классификации.}
    
    В парадигме P300 решается задача бинарной классификации отрезков ЭЭГ. Задача ~--- определить наличие в отрезке ЭЭГ потенциала P300.
    
    Для данных, рассмотренных в двух предыдущих разделах, требуется получить целевую функцию:
    $$g_{\theta}: \X \to \Y.$$
    
    Критерием качества в данной задаче является бинарная кросс-энтропия: 
    $$L =  -{\frac {1}{M}}\sum _{m=1}^{M}\ {\bigg [}y_{m}\log {p}_{m}+(1-y_{m})\log(1-{p}_{m}){\bigg ]}$$
    $$p_m = g_{\theta}(\X_m) \text{ ~--- вероятность 1 класса для } \X_m$$

    Решается оптимизационная задача:
    \begin{equation*}
    \hat{\theta} = \arg\max_{\theta} L(\theta, \X).
    \end{equation*}
    
    Внешними критериями качества для задачи классификации являются точность и F1-score.

\section{Вычислительный эксперимент}
\subsection{Постановка эксперимента}
Основная цель эксперимента~--- подобрать оптимальную структуру модели классификации на основе NODE с тезнорным представлением сигнала  с распространёнными на практике подходами к классификации сигналов.
    
    \paragraph{Постановка эксперимента.}
    Для запуска вычислительного эксперимента использовался набор данных, описанный в следующей секции. Для реализации алгоритмов была использована библиотека PyTorch \cite{pytorch}, для автоматизации процесса обучения~--- Lightning \cite{lightning}.
    
    Набор данных делился на две части: обучающую и валидационную выборки в соотношении 80:20. Для запусков экспериментов на нерегулярных данных производилась модификация исходных данных. Для каждой последовательности случайным образом удалялось 20\% точек. Таким образом, шаг по времени между точками становился нерегулярным. Для удобства вычислений длины всех последовательностей сохранялись одинаковыми, что также позволило получить качество классических моделей на нерегулярных данных. 
    
    Таким образом, для рассмотренного набора данных было обучено по три модели с использованием Neural ODE. Для набора данных потенциалов P300 также был обучен классификатор на римановской геометрии и свёрточная нейросеть для сравнения качества.
    
    
    \paragraph{Условия эксперимента.}
    Все модели обучались с помощью библиотеки Lightning. Для обучения моделей использовался алгоритм оптимизации Adam. Оптимизационная задача ставилась относительно критерия перекрёстной энтропии. Параметры оптимизации и количество эпох подбирались экспериментально для каждой из моделей. Для каждой модели в процессе обучения сохранялись три лучших набора параметров, отобранные по F1-score на валидационной выборке.
    Все запуски обучения производились на CPU.

% \subsection{Экспериментальные данные}

%     Вычислительные эксперименты проводились на датасете потенциалов P300.

%         \paragraph{Участники.}
    
%             Набор данных записан на 60 здоровых людях (23 мужчины), никогда ранее не встречавшихся с нейроинтерфейсами. Возраст участников от 19 до 45, средний ~--- 28 лет. Все респонденты соответствовали необходимым условиям по здоровью и состоянию и подписали информированное согласие на эксперимент.
    
%         \paragraph{Стимулы и запись ЭЭГ.}
        
%             ЭЭГ записывались с помощью энцефалографа NVX-52 (МКС, Зеленоград) с частотой 500 Гц. Для записи использовались 8 губчатых электродов (Cz, P3, P4, PO3, POz, PO4, O1, O2), см. рис. \ref{electrodes}. Стимулы предъявлялись с помощью шлема HTC Vive Pro VR.
            
%             \begin{figure}[!h]
%                 \begin{minipage}[b]{0.39\textwidth}
%                 \centering
%                 \includegraphics[width=0.9\textwidth]{imgs/scalp.jpeg}
%                 \caption{Расположение электродов}
%                 \label{electrodes}
%                 \end{minipage}%
%                 \hfill
%                 \begin{minipage}[b]{0.56\textwidth}
%                 \centering
%                 \includegraphics[width=0.9\textwidth]{imgs/rvd_scene.jpeg}
%                 \caption{Сцена из VR игры с енотами и демонами}
%                 \label{gameplay}
%                 \end{minipage}
%             \end{figure}
            
%             Участникам эксперимента была предложена игра в виртуальной реальности с нейроуправлением, основанным на классификации потенциалов P300. Во время игры на этапе обучения производилась тренировка логистической регрессии, во время игры на этапе обратной связи НКИ отрезки ЭЭГ классифицировались с использованием обученной на участнике логистической регрессии. Нейрокомпьютерный интерфейс был встроен в игровой сюжет, в котором игрок должен был кормить енотов и защищать их от демонов.
                
%             %В дизайне игры используется oddball paradigm (неожиданные стимулы). 
%             На этапе обучения участнику предъявлялось пять упорядоченных визуальных стимулов ~--- анимированных енотов (см. рис.\ref{gameplay}). Активации стимулов ~--- прыжки енотов в случайном порядке. Случайный порядок введен для того, чтобы предотвратить ожидание активации. Обучение составляло пять игровых актов (подробная структура приведена в следующем параграфе). В каждом акте участникам указывался конкретный енот, на котором они должны были концентрироваться: к примеру, считать количество прыжков. Каждый акт состоял из 10 активаций каждой цели, после чего указанный в начале акта енот получал еду. По итогу этапа обучения получалось 50 целевых (содержащих потенциал P300) и 450 нецелевых отрезков ЭЭГ для обучения бинарного классификатора.
            
%             Во время этапа обратной связи НКИ, стимулы предъявлялись как анимированные демоны.  Выбранный демон, при достаточной фокусировке игрока, уничтожался в конце акта, оставшиеся в живых  ~--- подходили ближе. Как только демон подходил достаточно близко к игроку, он хватал ближайшего к нему енота и убегал. В любом случае после исчезновения с экрана одного демона, на его месте, в начальной позиции, появлялся новый демон, чтобы количество стимулов оставалось постоянным.
            
%             Все данные электроэнцефалограмм, полученных во время игры были записаны для дальнейшего анализа. 
    
%         \paragraph{Термины и структура данных.}
        
%             Введём термины, используемые в дальнейшем:
           
%             \begin{itemize}
%                 \item Эпоха ~--- секундный отрезок ЭЭГ, отсчитывающийся после активации стимула.
                
%                 \item Раунд (блок) ~--- последовательность эпох с общим целевым стимулом, где каждый стимул активируется один раз, порождая одну эпоху. Активация стимулов происходит в случайном порядке.
        
%                 \item Акт ~--- последовательность раундов с общим целевым стимулом, после которого принимается решение о целевом стимуле. Количество актов в экспериментах ~--- 5 при обучении и 10 при игре.
        
%                 \item Запись (игра) ~--- последовательность актов, полученная от одного человека за одну сессию записи (одну игру).
                
%                 \item Датасет ~--- коллекция записей, сделанных на одном и том же оборудовании с одинаковыми визуальными активациями.
%             \end{itemize}
        
            
            
%             Пусть $E$ ~---  количество каналов записи ЭЭГ, $N$ ~---  число отсчётов времени в одной эпохе.
        
%             Обычно в парадигме P300 представлено несколько стимулов (в рассматриваемом датасете ~--- 5) и основной задачей является определение стимула, выбранного пользователем ~--- задача мультиклассовой классификации.
            
%             На данный момент мультиклассовая задача решается как агрегация бинарных задач, т.е.:
%             $$P^{mult}_j = \frac{\sum_{i=1}^MP^{bin}_{j,i}}{M}$$
%             $j$ ~--- рассматриваемый класс мультиклассовой классификации,\\
%             $M$ ~--- количество предъявлений одного стимула в акте игры,\\
%             $s$ ~--- число визуальных стимулов, количество классов мультиклассовой классификации,\\
%             $P_{j, i}^{bin}$ ~--- вероятность того, что стимул $j$ целевой при предъявлении $i$.\\
            
%             Итоговый стимул выбирается как стимул с наибольшей вероятностью $P^{mult}_j$. 
            
%             В данной работе мы рассматриваем решение бинарной задачи, описанной в разделе <<Постановка задачи>>. В данном наборе данных как обучающая, так и тестовая выборки являются несбалансированными, так как каждый стимул активируется одинаковое количество раз, но только один из них является целевым. Таким образом, получается соотношение классов  $1$ к $(s-1)$, где $s$ ~--- количество стимулов.
            
%             Итоговое количество записей в датасете: 56540.
            
%             Данные можно получить по \href{https://gin.g-node.org/v-goncharenko/neiry-demons/raw/master/nery_demons_dataset.zip}{ссылке} или воспользоваться датасетом DemonsP300 библиотеки \href{https://github.com/NeuroTechX/moabb}{moabb} \cite{moabb}.

\addcontentsline{toc}{section}{\protect\numberline{}Список литературы}
\bibliographystyle{ugost2008}
\bibliography{references}

\end{document}
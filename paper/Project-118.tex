\documentclass[a4paper, 12pt]{article} %{article}
\usepackage{arxiv}

\usepackage[utf8]{inputenc}
\usepackage[english, russian]{babel}
\usepackage[T2A]{fontenc}
\usepackage{url}
\usepackage{booktabs}
\usepackage{amsfonts}
\usepackage{nicefrac}
\usepackage{microtype}
\usepackage{lipsum}
\usepackage{graphicx}
\usepackage{natbib}
\usepackage{doi}
\renewcommand{\abstractname}{Аннотация}


\title{Непрерывное время при построении нейроинтерфейса BCI.}

\author{ Соболевский Федор\\
	Кафедра интеллектуальных систем \\
        ФПМИ МФТИ\\ 
	\texttt{sobolevskii.fa@phystech.edu} 
	\AND
        Консультант: Самохина Алина\\
	Кафедра интеллектуальных систем \\
        ФПМИ МФТИ\\
	\texttt{alina.samokhina@phystech.edu} 
        \AND
        Эксперт: д.ф.-м.н. Стрижов Вадим\\
	Вычислительный центр им. А.\,А.\,Дородницына\\
        ФИЦ ИУ РАН \\
	\texttt{strijov@phystech.edu}
	%% Coauthor \\
	%% Affiliation \\
	%% Address \\
	%% \texttt{email} \\
	%% \And
	%% Coauthor \\
	%% Affiliation \\
	%% Address \\
	%% \texttt{email} \\
	%% \And
	%% Coauthor \\
	%% Affiliation \\
	%% Address \\
	%% \texttt{email} \\
}
\date{\today}

%\renewcommand{\shorttitle}{\textit{arXiv} Template}

%%% Add PDF metadata to help others organize their library
%%% Once the PDF is generated, you can check the metadata with
%%% $ pdfinfo template.pdf
\hypersetup{
pdftitle={A template for the arxiv style},
pdfsubject={q-bio.NC, q-bio.QM},
pdfauthor={David S.~Hippocampus, Elias D.~Striatum},
pdfkeywords={First keyword, Second keyword, More},
}

\begin{document}
\maketitle

\begin{abstract}
В задачах декодирования сигналов входные данные представляют собой одномерные или многомерные временные ряды. Применение методов, основанных на нейронных обыкновенных дифференциальных уравнениях, позволяет работать с временными рядами как с непрерывными по времени. Недавние исследования показывают, что подобные методы могут давать заметно более точные по метрикам качества результаты в задачах классификации сигналов, чем методы, работающие с дискретным представлением временных рядов. В данной работе рассматриваются различные методы, основанные на непрерывном представлении временных рядов, в приложении к задаче классификации электроэнцефалограмм (ЭЭГ) и аппроксимации исходного сигнала. В предложенном подходе к построению модели машинного обучения предполагается работа с функциональным пространством сигнала вместо его дискретного представления и использование пространства параметров аппроксимируемой функции в качестве признакового. Основной результат работы – построение обратимого потока и подбор оптимальных размерностей на каждом слое нейросети. 
\end{abstract}


\keywords{EEG \and continuos time series \and neural ODE \and neural CDE}

% \section{Введение}
 
\bibliographystyle{unsrtnat}
\bibliography{references}

\end{document}

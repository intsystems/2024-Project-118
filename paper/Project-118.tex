\documentclass[a4paper, 12pt]{article} %{article}
\usepackage{arxiv}

\usepackage[utf8]{inputenc}
\usepackage[english, russian]{babel}
\usepackage[T2A]{fontenc}
\usepackage{url}
\usepackage{booktabs}
\usepackage{amsfonts}
\usepackage{nicefrac}
\usepackage{microtype}
\usepackage{lipsum}
\usepackage{graphicx}
\usepackage{natbib}
\usepackage{doi}
\renewcommand{\abstractname}{Аннотация}

\newcommand\argmin{\mathop{\arg\min}}
\newcommand{\T}{^{\text{\tiny\sffamily\upshape\mdseries T}}}
\newcommand{\hchi}{\hat{\boldsymbol{\chi}}}
\newcommand{\hphi}{\hat{\boldsymbol{\varphi}}}
\newcommand{\bchi}{\boldsymbol{\chi}}
\newcommand{\A}{\mathbf{A}}
\newcommand{\bb}{\mathbf{b}}
\newcommand{\B}{\mathcal{B}}
\newcommand{\W}{\mathbf{W}}
\newcommand{\E}{\mathbf{E}}
\newcommand{\V}{\mathbb{V}}
\renewcommand{\U}{\mathbb{U}}
\newcommand{\x}{\mathbf{x}}
\newcommand{\y}{\mathbf{y}}
\newcommand{\Y}{\mathbf{Y}}
\newcommand{\X}{\mathbf{X}}
\newcommand{\Z}{\mathbf{Z}}
\newcommand{\hx}{\hat{x}}
\newcommand{\hX}{\hat{\X}}
\newcommand{\hy}{\hat{y}}
\newcommand{\M}{\mathcal{M}}
\newcommand{\N}{\mathcal{N}}
\newcommand{\R}{\mathbb{R}}
\newcommand{\p}{p(\cdot)}
\newcommand{\cc}{\mathbf{c}}
\newcommand{\m}{\mathbf{m}}
\newcommand{\bt}{\mathbf{t}}
\newcommand{\e}{\mathbf{e}}
\newcommand{\h}{\mathbf{h}}
\newcommand{\q}{q(\cdot)}
\newcommand{\uu}{\mathbf{u}}
\newcommand{\vv}{\mathbf{v}}
\newcommand{\dd}{\partial}


\title{Непрерывное время при построении нейроинтерфейса BCI.}

\author{ Соболевский Федор\\
	Кафедра интеллектуальных систем \\
        ФПМИ МФТИ\\ 
	\texttt{sobolevskii.fa@phystech.edu} 
	\AND
        Консультант: Самохина Алина\\
	Кафедра интеллектуальных систем \\
        ФПМИ МФТИ\\
	\texttt{alina.samokhina@phystech.edu} 
        \AND
        Эксперт: д.ф.-м.н. Стрижов Вадим\\
	Вычислительный центр им. А.\,А.\,Дородницына\\
        ФИЦ ИУ РАН \\
	\texttt{strijov@phystech.edu}
	%% Coauthor \\
	%% Affiliation \\
	%% Address \\
	%% \texttt{email} \\
	%% \And
	%% Coauthor \\
	%% Affiliation \\
	%% Address \\
	%% \texttt{email} \\
	%% \And
	%% Coauthor \\
	%% Affiliation \\
	%% Address \\
	%% \texttt{email} \\
}
\date{\today}

%\renewcommand{\shorttitle}{\textit{arXiv} Template}

%%% Add PDF metadata to help others organize their library
%%% Once the PDF is generated, you can check the metadata with
%%% $ pdfinfo template.pdf
\hypersetup{
pdftitle={A template for the arxiv style},
pdfsubject={q-bio.NC, q-bio.QM},
pdfauthor={David S.~Hippocampus, Elias D.~Striatum},
pdfkeywords={First keyword, Second keyword, More},
}

\begin{document}
\maketitle


\begin{abstract}
В задачах декодирования сигналов входные данные представляют собой одномерные или многомерные временные ряды. Применение методов, основанных на нейронных обыкновенных дифференциальных уравнениях, позволяет работать с временными рядами как с непрерывными по времени. Недавние исследования показывают, что подобные методы могут давать заметно более точные по метрикам качества результаты в задачах классификации сигналов, чем методы, работающие с дискретным представлением временных рядов. В данной работе рассматриваются различные методы, основанные на непрерывном представлении временных рядов, в приложении к задаче классификации электроэнцефалограмм (ЭЭГ) и аппроксимации исходного сигнала. В предложенном подходе к построению модели машинного обучения предполагается работа с функциональным пространством сигнала вместо его дискретного представления и использование пространства параметров аппроксимируемой функции в качестве признакового. Основной результат работы – построение обратимого потока и подбор оптимальных размерностей на каждом слое нейросети. 
\end{abstract}


\keywords{EEG \and continuous time series \and neural ODE \and neural CDE}


\section{Введение}
Данная работа посвящена декодированию сигналов~--- одномерных или многомерных временных рядов. Примерами актуальных задач декодирования сигналов являются задачи расшифровки сигналов акселерометров, пульсометров, электроэнцефалограмм (ЭЭГ) и электрокортикограмм (ЭКоГ). Многие сигналы, с которыми приходится работать в практических задачах, по природе своей непрерывны по времени, однако наиболее распространенный подход к расшифровке сигнала~--- работа с дискретизированной версией сигнала, то есть работа с ним, как с дискретным временным рядом. Основной проблемой при применении такого подхода является нерегулярность данных: данные могут иметь неравные отсчёты по времени или пропуски, что несколько усложняет их обработку.

В ряде недавних работ по обработке временных рядов были предложены новые методы, позволяющие работать с временными рядами как с непрерывными по времени. Таковыми являются, например, методы, основанные на применении \textit{нейронных обыкновенных дифференциальных уравнениях} (НОДУ) в качестве скрытого слоя рекурентной нейронной сети. В работе <...> был предложен подход, основанный на применении \textit{нейронных управляемых дифференциальных уравнений}, позволяющий устранить однозначную зависимость решения дифференциального уравнения от начальных данных, с целью обеспечить возможность корректировки предсказания нейросети на основе данных, поступающих после исходного предсказания. Другие методы, работающие с непрерывным представлением сигналов, включают в себя нейронные стохастические дифференциальные уравнения, алгоритмы S4, S5, D4 и другие.

Подходы, работающие с непрерывными по времени данными, предположительно могут устранить проблемы, связанные с нерегулярностью данных, так как любая нерегулярность может быть устранена интерполяцией. В некоторых задачах декодирования сигналов методы, основанные на НОДУ, и другие методы, работающие с непрерывным представлением сигнала, показали большую точность, чем методы декодирования дискретных последовательностей. В данной работе фокусом являются задачи регрессии~--- восстановления непрерывной по времени последовательности по непрерывному сигналу.

В качестве основной задачи декодирования сигнала в данной работе рассматривается классификация ЭКоГ. В последнее время большое количество работ посвящено методам считывания мозговой активности и декодирования информации. Основным приложением данных методов являются нейрокомпьютерные интерфейсы (НКИ)~--- технология, позволяющая человеку взаимодействовать с компьютером с помощью анализа данных о мозговой активности. Подобные интерфейсы могут использоваться как в медицинских целях, так и в реакреационных. Вполне естественно, что для введения устройств, считывающих и расшифровывающих мозговую активность, в эксплуатацию необходимо, во-первых, достичь высокой точности декодирования сигнала, и, во-вторых, достичь оптимальной сложности модели по количеству параметров.

Целью данной работы является исследование алгоритмов, основанных на НОДУ, в задаче декодирования ЭКоГ для последующего восстановления непрерывной зависимости движения конечности, и сравнение точности по стандартной метрике MSE с результатами, полученными методами, работающими с дискретным представлением времени. Для этого указанные модели, наряду с современными моделями, работающими с дискретным представлением времени, применяются на выборке Epidural-ECoG Food-Tracking. Данная выборка содержит данные 8 экспериментов с обезьяной, в которых синхронно считывались сигналы с ЭКоГ и положение 6 точек руки обезьяны.

Применение методов, основанных на НОДУ, позволяет работать с ЭКоГ как с непрерывными по времени, что предоставляет возможности дальнейшего исследования непрерывных сигналов, например, в контексте построения фазовых пространств для непрерывного времени и/или пространства. Применимость вышеуказанных методов для решения актуальных задач декодирвания сигналов может послужить мотивацией для развития теории моделей машинного обучения, работающих с непрерывными объектными пространствами.


\section{Постановка задачи}
В данной работе мы имеем дело с сигналами в их непрерывном представлении. Пусть имеется непрерывный процесс (например, активность мозга, движение конечности):
$$V(t), t \in \R$$
Тогда данные выборки, регистрируемый сигнал~--- реализация процесса $V(t)$:
$$\X = \{\x_t\}_{t\in T},\  \x_t \in \R^E,\  T = \{t_i\}_{i=1}^{N}, t_i \in \R$$ 
$$\x_{t_i} \approx V(t_i).$$

Для непрерывного и дискретного по времени сигналов решается задачи регрессии. Пусть в выборке присутствует $M$ наблюдений. Тогда исходный и целевой сигналы определяются следующим образом:
$$\X = \{\X_i\}_{i=1}^{M},$$
$$\X_i = \{\x_t\}_{t\in T}, \ \x_t \in \R^E, \ T = \{t_i\}_{i=1}^{N}, t_i \in \R$$ 

$$\Y = \{\Y_i\}_{i=1}^{M},$$
$$\Y_i = \{\y_t\}_{t\in T}, \ \y_t \in \R^F.$$ 
Для таких данных постановка задачи регрессии следующая: требуется получить целевую функцию
$$g_\theta: \X \to \Y.$$

В непрерывном представлении целевой сигнал принимает вид, подобный исходному:
$$U(t), t\in\R, \ \y_{t_i} \approx U(t_i).$$
Тогда целевая функция есть отображение между двумя пространствами вектор-функций $\V$ и $\U$:
$$g_\theta: \V \to \U.$$

В обоих случаях критерием качества является отклонение предсказаний модели от известных значений на тестовых данных по стандартной метрике MSE:
$$L = \frac{1}{n}\sum\limits_{i=1}^{M}\left((g_\theta(\X))_{t_i} - \y_{t_i}\right)^2.$$

Решается оптимизационная задача:
$$\hat{\theta} = \arg\min_{\theta} L(\theta, \X).$$

\bibliographystyle{unsrtnat}
\bibliography{references}

\end{document}